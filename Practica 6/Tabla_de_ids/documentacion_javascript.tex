\documentclass[12pt,a4paper]{article}
\usepackage[utf8]{inputenc}
\usepackage[spanish]{babel}
\usepackage{geometry}
\usepackage{longtable}
\usepackage{booktabs}
\usepackage{xcolor}
\usepackage{colortbl}
\usepackage{array}

\geometry{margin=2cm}

\title{\textbf{Documentación de Funciones JavaScript e IDs\\Proyecto RiseBook - APPS\_WEB}}
\author{Baz Balderas}
\date{\today}

\begin{document}

\maketitle

\section{Introducción}
Este documento contiene la documentación completa de todas las funciones de JavaScript y los identificadores (IDs) utilizados en el proyecto RiseBook, organizado por prácticas del 3 al 6.

\section{Funciones de JavaScript Utilizadas}

\begin{longtable}{|p{3cm}|p{4cm}|p{8cm}|}
\hline
\rowcolor{lightgray}
\textbf{Función/Método} & \textbf{Práctica} & \textbf{Uso y Descripción} \\
\hline

\texttt{document.getElementById()} & Práctica 3, 4, 5, 6 & Busca un elemento HTML por su ID único. Ejemplo: \texttt{document.getElementById('guardar')} obtiene el botón con ID "guardar" \\
\hline

\texttt{document.querySelectorAll()} & Práctica 3, 5 & Busca todos los elementos que coincidan con un selector CSS. Ejemplo: \texttt{document.querySelectorAll('textarea')} obtiene todas las cajas de texto \\
\hline

\texttt{document.querySelector()} & Práctica 3, 5, 6 & Busca el primer elemento que coincida con un selector CSS. Ejemplo: \texttt{document.querySelector('\#tablaTareas tbody')} busca el cuerpo de la tabla \\
\hline

\texttt{.innerHTML} & Práctica 3, 5, 6 & Cambia o obtiene el contenido HTML interno de un elemento. Se usa para llenar tablas dinámicamente \\
\hline

\texttt{.textContent} & Práctica 3, 4, 5, 6 & Cambia o obtiene solo el texto de un elemento, sin HTML. Se usa para mostrar mensajes y contadores \\
\hline

\texttt{.value} & Práctica 3, 4, 5, 6 & Obtiene o establece el valor de inputs, textareas y selects. Se usa para obtener datos del formulario \\
\hline

\texttt{.forEach()} & Práctica 3, 5, 6 & Recorre cada elemento de un array ejecutando una función para cada uno. Se usa para mostrar listas de datos \\
\hline

\texttt{.push()} & Práctica 3, 5, 6 & Agrega un nuevo elemento al final de un array. Se usa para añadir estudiantes, tareas y proyectos \\
\hline

\texttt{.splice()} & Práctica 3, 5, 6 & Elimina elementos de un array en posiciones específicas. Se usa para eliminar registros \\
\hline

\texttt{.addEventListener()} & Práctica 3, 4, 5, 6 & Conecta eventos (click, load, etc.) con funciones. Se usa para manejar interacciones del usuario \\
\hline

\texttt{document.addEventListener()} & Práctica 3, 5, 6 & Escucha eventos en todo el documento. Se usa principalmente con 'DOMContentLoaded' \\
\hline

\texttt{.find()} & Práctica 6 & Busca el primer elemento de un array que cumpla una condición. Se usa para buscar proyectos por ID \\
\hline

\texttt{.findIndex()} & Práctica 5, 6 & Busca la posición de un elemento en un array que cumpla una condición. Se usa para localizar elementos \\
\hline

\texttt{localStorage.setItem()} & Práctica 5, 6 & Guarda datos en el almacenamiento local del navegador. Se usa para mantener la sesión del usuario \\
\hline

\texttt{localStorage.getItem()} & Práctica 5, 6 & Recupera datos del almacenamiento local. Se usa para obtener información de la sesión \\
\hline

\texttt{localStorage.removeItem()} & Práctica 5, 6 & Elimina datos del almacenamiento local. Se usa al cerrar sesión \\
\hline

\texttt{JSON.stringify()} & Práctica 5, 6 & Convierte un objeto JavaScript a texto JSON. Se usa para guardar objetos en localStorage \\
\hline

\texttt{JSON.parse()} & Práctica 5, 6 & Convierte texto JSON a objeto JavaScript. Se usa para recuperar objetos de localStorage \\
\hline

\texttt{.trim()} & Práctica 5, 6 & Elimina espacios en blanco al inicio y final de un string. Se usa para validar campos de formulario \\
\hline

\texttt{.test()} & Práctica 4, 5, 6 & Prueba si un string coincide con una expresión regular. Se usa para validar formato de email \\
\hline

\texttt{.split()} & Práctica 5, 6 & Divide un string en un array usando un separador. Se usa para extraer el nombre del email \\
\hline

\texttt{setTimeout()} & Práctica 5, 6 & Ejecuta una función después de un tiempo determinado. Se usa para redirigir después del login \\
\hline

\texttt{window.location.href} & Práctica 5, 6 & Cambia la URL actual, redirigiendo a otra página. Se usa para navegar entre páginas \\
\hline

\texttt{.preventDefault()} & Práctica 3, 4, 5, 6 & Evita el comportamiento por defecto de un evento. Se usa para evitar que formularios recarguen la página \\
\hline

\texttt{.length} & Práctica 3, 5, 6 & Propiedad que indica el número de elementos en un array o caracteres en un string \\
\hline

\texttt{.getAttribute()} & Práctica 6 & Obtiene el valor de un atributo HTML. Se usa para obtener data-proyecto-id \\
\hline

\texttt{.setAttribute()} & Práctica 6 & Establece el valor de un atributo HTML. Se usa para guardar data-proyecto-id \\
\hline

\texttt{.removeAttribute()} & Práctica 6 & Elimina un atributo HTML. Se usa para limpiar data-proyecto-id \\
\hline

\texttt{.matches()} & Práctica 6 & Verifica si un elemento coincide con un selector CSS. Se usa para eventos de click \\
\hline

\texttt{.selectedIndex} & Práctica 6 & Obtiene o establece la opción seleccionada en un select. Se usa para resetear formularios \\
\hline

\texttt{new Date()} & Práctica 5, 6 & Crea un objeto de fecha actual. Se usa para timestamps y fechas de inicio \\
\hline

\texttt{Date.now()} & Práctica 6 & Obtiene la fecha actual en milisegundos. Se usa para generar IDs únicos \\
\hline

\texttt{Math.random()} & Práctica 6 & Genera un número aleatorio entre 0 y 1. Se usa para generar IDs únicos \\
\hline

\texttt{.toString()} & Práctica 6 & Convierte un número a string con una base específica. Se usa para generar IDs alfanuméricos \\
\hline

\texttt{.slice()} & Práctica 6 & Extrae una parte de un string o array. Se usa para acortar IDs generados \\
\hline

\texttt{console.log()} & Práctica 5, 6 & Muestra información en la consola del navegador. Se usa para debugging \\
\hline

\texttt{alert()} & Práctica 6 & Muestra una ventana emergente con un mensaje. Se usa para alertas al usuario \\
\hline

\end{longtable}

\newpage

\section{Identificadores (IDs) Utilizados en el Proyecto}

\subsection{Práctica 3 - Sistema de Gestión de Estudiantes (SIGEA)}

\begin{longtable}{|p{4cm}|p{3cm}|p{8cm}|}
\hline
\rowcolor{lightgray}
\textbf{ID} & \textbf{Elemento} & \textbf{Uso en JavaScript} \\
\hline

\texttt{dropdownCarrera} & Button & \texttt{document.getElementById('dropdownCarrera').textContent = carrera} - Actualiza el texto del botón de carrera seleccionada \\
\hline

\texttt{guardar} & Button & \texttt{document.getElementById('guardar').addEventListener('click', guardarEstudiante)} - Conecta el botón con la función de guardar \\
\hline

\texttt{limpiar} & Button & \texttt{document.getElementById('limpiar').addEventListener('click', limpiarFormulario)} - Conecta el botón con la función de limpiar \\
\hline

\texttt{tablaEstudiantes} & Table & \texttt{document.querySelector('\#tablaEstudiantes tbody')} - Accede al cuerpo de la tabla para llenarla con estudiantes \\
\hline

\end{longtable}

\subsection{Práctica 4 - Sistema de Login Básico}

\begin{longtable}{|p{4cm}|p{3cm}|p{8cm}|}
\hline
\rowcolor{lightgray}
\textbf{ID} & \textbf{Elemento} & \textbf{Uso en JavaScript} \\
\hline

\texttt{btnInicioSesion} & Button & \texttt{document.getElementById('btnInicioSesion')} - Obtiene el botón de ingreso para agregar el evento click \\
\hline

\texttt{mensajito} & H3 & \texttt{document.getElementById('mensajito')} - Muestra mensajes de éxito o error al usuario \\
\hline

\texttt{email} & Textarea & \texttt{document.getElementById('email').value} - Obtiene el email ingresado por el usuario \\
\hline

\texttt{password} & Textarea & \texttt{document.getElementById('password').value} - Obtiene la contraseña ingresada por el usuario \\
\hline

\texttt{registro} & Div & Usado solo para estilos CSS, no manipulado en JavaScript \\
\hline

\texttt{promocionsuperior} & Div & Usado solo para estilos CSS, no manipulado en JavaScript \\
\hline

\texttt{logo} & Image & Usado solo para estilos CSS, no manipulado en JavaScript \\
\hline

\texttt{promocioninferior} & Div & Usado solo para estilos CSS, no manipulado en JavaScript \\
\hline

\texttt{mexicolindo} & Image & Usado solo para estilos CSS, no manipulado en JavaScript \\
\hline

\end{longtable}

\clearpage

\subsection{Práctica 5 - Sistema de Login con Dashboard de Tareas}

\begin{longtable}{|p{4cm}|p{3cm}|p{8cm}|}
\hline
\rowcolor{lightgray}
\textbf{ID} & \textbf{Elemento} & \textbf{Uso en JavaScript} \\
\hline

\texttt{btnInicioSesion} & Button & \texttt{document.getElementById('btnInicioSesion')} - Maneja el evento de inicio de sesión \\
\hline

\texttt{mensajito} & H3 & \texttt{messageElement.textContent = 'mensaje'} - Muestra retroalimentación al usuario \\
\hline

\texttt{email} & Input & \texttt{document.getElementById('email').value} - Obtiene el email del formulario \\
\hline

\texttt{password} & Input & \texttt{document.getElementById('password').value} - Obtiene la contraseña del formulario \\
\hline

\texttt{nombreUsuario} & H1 & \texttt{document.getElementById('nombreUsuario').textContent = `Dashboard - \${usuario.nombre}`} - Muestra el nombre del usuario logueado \\
\hline

\texttt{tablaTareas} & Table & \texttt{document.querySelector('\#tablaTareas tbody')} - Accede al cuerpo de la tabla de tareas \\
\hline

\texttt{nombreTarea} & Input & \texttt{document.getElementById('nombreTarea').value} - Obtiene el nombre de la nueva tarea \\
\hline

\texttt{descripcionTarea} & Input & \texttt{document.getElementById('descripcionTarea').value} - Obtiene la descripción de la nueva tarea \\
\hline

\texttt{contadorTareas} & Span & \texttt{document.getElementById('contadorTareas').textContent = `Tareas pendientes: \${tareas.length}`} - Actualiza el contador de tareas \\
\hline

\texttt{btnAgregarTarea} & Button & \texttt{document.getElementById('btnAgregarTarea').addEventListener('click', agregarTarea)} - Conecta el botón con la función \\
\hline

\texttt{btnCerrarSesion} & Button & \texttt{document.getElementById('btnCerrarSesion').addEventListener('click', cerrarSesion)} - Maneja el cierre de sesión \\
\hline

\end{longtable}

\subsection{Práctica 6 - Sistema Completo de Gestión de Proyectos y Tareas}

\begin{longtable}{|p{4cm}|p{3cm}|p{8cm}|}
\hline
\rowcolor{lightgray}
\textbf{ID} & \textbf{Elemento} & \textbf{Uso en JavaScript} \\
\hline

\texttt{btnInicioSesion} & Button & \texttt{document.getElementById('btnInicioSesion')} - Maneja el inicio de sesión del usuario \\
\hline

\texttt{mensajito} & H3 & \texttt{messageElement.textContent = 'mensaje'} - Muestra mensajes de validación \\
\hline

\texttt{email} & Textarea & \texttt{document.getElementById('email').value} - Captura el email del usuario \\
\hline

\texttt{password} & Textarea & \texttt{document.getElementById('password').value} - Captura la contraseña del usuario \\
\hline

\texttt{dashboard-body} & Body & Usado para estilos CSS del dashboard \\
\hline

\texttt{dashboard-header} & Div & Contenedor del header, usado para estilos \\
\hline

\texttt{logo-dashboard} & Div & Contenedor del logo, usado para estilos \\
\hline

\texttt{logo-pequeño} & Image & Logo del dashboard, usado para estilos \\
\hline

\texttt{nombreUsuario} & H1 & \texttt{document.getElementById('nombreUsuario')} - Muestra el nombre del usuario logueado \\
\hline

\texttt{btnCerrarSesion} & Button & \texttt{document.getElementById('btnCerrarSesion')?.addEventListener('click', cerrarSesion)} - Maneja el cierre de sesión \\
\hline

\texttt{contador-container} & Div & Contenedor del contador, usado para estilos \\
\hline

\texttt{contadorProyectos} & H2 & \texttt{document.getElementById('contadorProyectos').textContent = `Proyectos creados: \${proyectos.length}`} - Muestra el número de proyectos \\
\hline

\texttt{form-proyectos} & Div & Contenedor del formulario de proyectos \\
\hline

\texttt{nombreProyecto} & Input & \texttt{document.getElementById('nombreProyecto').value.trim()} - Obtiene el nombre del nuevo proyecto \\
\hline

\texttt{descripcionProyecto} & Input & \texttt{document.getElementById('descripcionProyecto').value.trim()} - Obtiene la descripción del proyecto \\
\hline

\texttt{fechaFinProyecto} & Input & \texttt{document.getElementById('fechaFinProyecto').value.trim()} - Obtiene la fecha fin del proyecto \\
\hline

\texttt{btnCrearProyecto} & Button & \texttt{document.getElementById('btnCrearProyecto')?.addEventListener('click', agregarProyecto)} - Crea nuevos proyectos \\
\hline

\texttt{form-tareas} & Div & Contenedor del formulario de tareas \\
\hline

\texttt{nombreTarea} & Input & \texttt{document.getElementById('nombreTarea').value.trim()} - Obtiene el nombre de la nueva tarea \\
\hline

\texttt{descripcionTarea} & Input & \texttt{document.getElementById('descripcionTarea').value.trim()} - Obtiene la descripción de la tarea \\
\hline

\texttt{estadoTarea} & Select & \texttt{document.getElementById('estadoTarea').value} - Obtiene el estado seleccionado de la tarea \\
\hline

\texttt{prioridadTarea} & Select & \texttt{document.getElementById('prioridadTarea').value} - Obtiene la prioridad seleccionada \\
\hline

\texttt{fechaVencimientoTarea} & Input & \texttt{document.getElementById('fechaVencimientoTarea').value.trim()} - Obtiene la fecha de vencimiento \\
\hline

\texttt{proyectoSeleccionado} & Button & \texttt{document.getElementById('proyectoSeleccionado')} - Botón del menú desplegable de proyectos \\
\hline

\texttt{btnAgregarTarea} & Button & \texttt{document.getElementById('btnAgregarTarea')?.addEventListener('click', agregarTarea)} - Agrega nuevas tareas \\
\hline

\texttt{tabla-proyectos-container} & Div & Contenedor de la tabla de proyectos \\
\hline

\texttt{tablaProyectos} & Table & \texttt{document.querySelector('\#tablaProyectos tbody')} - Muestra la lista de proyectos \\
\hline

\texttt{tabla-tareas-container} & Div & Contenedor de la tabla de tareas \\
\hline

\texttt{tituloTareas} & H2 & \texttt{document.getElementById('tituloTareas').textContent = `Tareas del proyecto: \${proyecto.nombre}`} - Muestra el proyecto actual \\
\hline

\texttt{tablaTareas} & Table & \texttt{document.querySelector('\#tablaTareas tbody')} - Muestra las tareas del proyecto seleccionado \\
\hline

\end{longtable}

\section{Resumen de Funcionalidades}

\subsection{Métodos de Manipulación del DOM}
\begin{itemize}
\item \textbf{Selección de elementos:} \texttt{getElementById()}, \texttt{querySelector()}, \texttt{querySelectorAll()}
\item \textbf{Modificación de contenido:} \texttt{innerHTML}, \texttt{textContent}, \texttt{value}
\item \textbf{Manejo de atributos:} \texttt{getAttribute()}, \texttt{setAttribute()}, \texttt{removeAttribute()}
\end{itemize}

\subsection{Métodos de Arrays}
\begin{itemize}
\item \textbf{Modificación:} \texttt{push()}, \texttt{splice()}
\item \textbf{Búsqueda:} \texttt{find()}, \texttt{findIndex()}
\item \textbf{Iteración:} \texttt{forEach()}
\end{itemize}

\subsection{Almacenamiento Local}
\begin{itemize}
\item \textbf{Gestión:} \texttt{localStorage.setItem()}, \texttt{localStorage.getItem()}, \texttt{localStorage.removeItem()}
\item \textbf{Serialización:} \texttt{JSON.stringify()}, \texttt{JSON.parse()}
\end{itemize}

\subsection{Eventos y Navegación}
\begin{itemize}
\item \textbf{Eventos:} \texttt{addEventListener()}, \texttt{preventDefault()}
\item \textbf{Navegación:} \texttt{window.location.href}
\item \textbf{Temporizadores:} \texttt{setTimeout()}
\end{itemize}

\end{document}
